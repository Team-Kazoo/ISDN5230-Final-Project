\chapter{Hardware Implementation}\label{chap:hardware}

%% =============================================================================
%% WRITING INSTRUCTIONS FOR YOUR COLLEAGUE
%% =============================================================================
%% This chapter describes the custom audio acquisition hardware (ESP32-C3 PCB).
%% Target length: 8-12 pages
%%
%% KEY POINTS TO COVER:
%% - Why custom hardware? (browser audio limitations)
%% - System architecture (MEMS → Codec → MCU → USB)
%% - Firmware design and implementation
%% - Web Serial API integration
%% - Performance comparison with browser-native audio
%%
%% WRITING STYLE:
%% - Match the technical depth of Chapter 3
%% - Use active voice: "We designed...", "The firmware implements..."
%% - Explain design rationale, not just configuration
%% - Include code snippets (10-20 lines max per example)
%% - Compare with alternatives where relevant
%%
%% REQUIRED DATA:
%% See the detailed writing guide document for the complete checklist
%% =============================================================================

\section{Motivation and Design Rationale}\label{sec:hw_motivation}

%% TODO: Write this section (target: 1.5-2 pages)
%%
%% OUTLINE:
%% 1. Browser audio limitations (3 main points):
%%    - Latency unpredictability (varies 40-80ms)
%%    - Lack of low-level control (sample format, buffer sizes)
%%    - Browser implementation differences (Chrome vs Firefox vs Safari)
%%
%% 2. Why custom hardware matters for our application:
%%    - Real-time performance requirements (sub-100ms end-to-end)
%%    - Healthcare/therapeutic use cases need consistency
%%    - Research reproducibility across different deployments
%%
%% 3. Design goals and requirements:
%%    - End-to-end latency: < 20ms (mic to AudioWorklet)
%%    - Sample rate: 44.1kHz, 16-bit resolution
%%    - Plug-and-play: USB, no driver installation
%%    - Cost-effective: BOM target < $15
%%    - Compact: pocket-sized form factor
%%
%% SUGGESTED STRUCTURE:
%% Paragraph 1: Problem statement - browser audio limitations
%% Paragraph 2: Impact on our vocal synthesis application
%% Paragraph 3: How hardware addresses these challenges
%% Paragraph 4: Design goals (can use itemized list)
%% Paragraph 5: Overview of our solution approach
%%
%% EXAMPLE OPENING:
%% "While the Web Audio API provides convenient access to microphone input
%% through the getUserMedia interface, this convenience comes with inherent
%% limitations that impact real-time audio applications..."

[WRITE YOUR CONTENT HERE]


\section{Hardware Architecture and Component Selection}\label{sec:hw_architecture}

%% TODO: Write this section (target: 2-3 pages)
%%
%% REQUIRED ELEMENTS:
%% 1. System block diagram (Figure ~\ref{fig:hw_architecture})
%%    MEMS Mic → I2S → ES8311 ADC → I2S → ESP32-C3 → USB → PC
%%
%% 2. Component descriptions with selection rationale:
%%
%%    MEMS Microphone:
%%    - Model number: [SPECIFY]
%%    - Key specs: SNR, sensitivity, frequency response
%%    - Why chosen: digital I2S output, low cost, small size
%%
%%    ES8311 Audio Codec:
%%    - 24-bit stereo ADC/DAC
%%    - Programmable gain, high-pass filter
%%    - Why chosen: ESP32 I2S compatibility, integrated features
%%
%%    ESP32-C3 Microcontroller:
%%    - RISC-V 32-bit @ 160MHz
%%    - 400KB SRAM, hardware I2S controller
%%    - Native USB 1.1 full-speed support
%%    - Why chosen: cost, performance, built-in USB, Arduino/ESP-IDF support
%%
%% 3. PCB design highlights:
%%    - Number of layers (2-layer recommended for cost)
%%    - Form factor and dimensions
%%    - Power supply architecture (USB → LDO → 3.3V)
%%    - I2S signal routing (impedance matching, trace length)
%%    - USB connector type (USB-C recommended)
%%
%% 4. Power consumption analysis (if measured):
%%    - Idle current
%%    - Active streaming current
%%    - Total power budget
%%
%% SUGGESTED FIGURES:
%% - Figure 4.1: System block diagram (REQUIRED)
%% - Figure 4.2: PCB photograph or 3D render (OPTIONAL but recommended)
%% - Figure 4.3: Signal flow timing diagram (OPTIONAL)
%%
%% CODE EXAMPLE (component initialization):
%% You can include brief snippets like ES8311 I2C configuration
%%
%% WRITING TIPS:
%% - For each component, explain: What it does → Why we chose it → Key specs
%% - Compare with alternatives briefly (e.g., why ESP32-C3 vs ESP32-S3?)
%% - Don't just list specifications; explain their significance
%%
%% EXAMPLE TEXT:
%% "The ESP32-C3 provides hardware I2S support through dedicated peripherals,
%% eliminating the timing uncertainty that software bit-banging would introduce.
%% Its native USB 1.1 interface enables direct connection to the host computer
%% without requiring external USB-to-serial converters..."

[WRITE YOUR CONTENT HERE]

\begin{figure}[htbp]
  \centering
  %% TODO: Insert your hardware block diagram here
  %% \includegraphics[width=0.9\textwidth]{figures/hardware_block_diagram.pdf}
  \fbox{\parbox{0.9\textwidth}{\centering\vspace{2cm}
  [INSERT HARDWARE BLOCK DIAGRAM HERE]\\
  MEMS Mic $\rightarrow$ ES8311 $\rightarrow$ ESP32-C3 $\rightarrow$ USB $\rightarrow$ Browser
  \vspace{2cm}}}
  \caption{Hardware system architecture showing signal flow from MEMS microphone through audio codec and microcontroller to USB interface. The ESP32-C3 implements I2S audio capture and USB CDC serial transmission, while the ES8311 codec performs analog-to-digital conversion with programmable gain and filtering.}
  \label{fig:hw_architecture}
\end{figure}


\section{ESP32-C3 Firmware Design}\label{sec:firmware}

%% TODO: Write this section (target: 3-4 pages)
%%
%% This is the most technical section. Include implementation details but
%% keep it at the right abstraction level.
%%
%% SUBSECTION STRUCTURE (you can use \subsection if needed):
%%
%% 3.1 Firmware Architecture Overview
%% - Three-layer architecture diagram
%% - Application layer: USB state machine, buffer management
%% - HAL/Driver layer: I2S driver, USB CDC driver, ES8311 config
%% - Hardware layer: I2S peripheral, USB peripheral
%%
%% 3.2 I2S Audio Capture Implementation
%% - I2S configuration parameters and rationale:
%%   * Mode: I2S_MODE_MASTER | I2S_MODE_RX
%%   * Sample rate: 44100 Hz (why?)
%%   * Bits per sample: 16 (why not 24?)
%%   * Channel: Mono vs Stereo configuration
%%   * DMA buffer size: [SPECIFY] samples (trade-off analysis)
%%
%% - Code example: I2S initialization
%% - Interrupt handling and DMA management
%% - Error handling and recovery
%%
%% 3.3 Audio Buffer Management
%% - Ring buffer architecture (why ring buffer?)
%% - Buffer size calculation (latency vs overflow prevention)
%% - Zero-copy optimizations where possible
%% - Flow control and backpressure handling
%%
%% 3.4 USB CDC Serial Implementation
%% - USB descriptor configuration
%% - Bulk transfer mode and packet sizes
%% - Custom framing protocol (if any):
%%   * Packet format: [HEADER | SAMPLE_COUNT | AUDIO_DATA | CRC?]
%%   * Synchronization mechanism
%% - Transfer scheduling to minimize latency
%%
%% 3.5 Latency Optimization Strategies
%% - Minimize unnecessary memory copies
%% - DMA buffer sizing trade-offs
%% - Interrupt priority configuration
%% - CPU frequency vs power consumption
%%
%% KEY CODE SNIPPETS TO INCLUDE (pseudo-code or simplified):
%% 1. I2S initialization structure (~15 lines)
%% 2. Main audio processing loop (pseudo-code, ~20 lines)
%% 3. USB transmission function signature and logic (simplified)
%%
%% EXAMPLE CODE BLOCK:
%% \begin{lstlisting}[language=C, caption={I2S configuration for audio capture}]
%% i2s_config_t i2s_config = {
%%     .mode = I2S_MODE_MASTER | I2S_MODE_RX,
%%     .sample_rate = 44100,
%%     .bits_per_sample = I2S_BITS_PER_SAMPLE_16BIT,
%%     .channel_format = I2S_CHANNEL_FMT_ONLY_LEFT,
%%     .communication_format = I2S_COMM_FORMAT_I2S,
%%     .dma_buf_count = 4,
%%     .dma_buf_len = 512,  // Trade-off: latency vs CPU overhead
%%     .use_apll = false,
%%     .intr_alloc_flags = ESP_INTR_FLAG_LEVEL1
%% };
%% \end{lstlisting}
%%
%% MEASUREMENTS TO INCLUDE:
%% - Measured CPU utilization during streaming
%% - Measured latency contribution of firmware processing
%% - Memory usage (heap, stack)
%%
%% WRITING TIPS:
%% - Explain WHY you made each configuration choice
%% - Discuss trade-offs (e.g., buffer size: smaller = lower latency but
%%   higher CPU overhead and risk of underrun)
%% - Compare with alternative approaches where relevant
%% - Keep code examples short and well-commented
%% - Use figures/diagrams for complex flows (state machines, data flow)

[WRITE YOUR CONTENT HERE]


\section{Web Serial API Integration}\label{sec:web_serial}

%% TODO: Write this section (target: 2-2.5 pages)
%%
%% This section bridges hardware (firmware) and software (web application).
%%
%% OUTLINE:
%%
%% 4.1 Web Serial API Overview
%% - W3C specification status
%% - Browser support (Chrome/Edge, not Firefox/Safari)
%% - Security model: user permission required
%% - Why Serial vs WebUSB? (simpler, CDC class = no custom driver)
%%
%% 4.2 Device Detection and Connection
%% - Requesting serial port with vendor/product ID filtering
%% - Opening port with configuration (baud rate is ignored for USB CDC)
%% - Error handling: device not found, permission denied
%%
%% JavaScript code example:
%% \begin{lstlisting}[language=JavaScript]
%% const port = await navigator.serial.requestPort({
%%     filters: [{ usbVendorId: 0x303A }]  // Espressif VID
%% });
%% await port.open({ baudRate: 921600 });  // Ignored for USB CDC
%% \end{lstlisting}
%%
%% 4.3 Audio Data Reception and Parsing
%% - Reading from serial port (streaming API)
%% - Parsing audio packets (match firmware framing)
%% - Converting Uint8Array to Int16Array for audio samples
%% - Handling incomplete packets and synchronization
%%
%% Code example: Reading loop
%%
%% 4.4 Integration with AudioWorklet Pipeline
%% - Architecture: Web Serial (Main Thread) → MessagePort → AudioWorklet
%% - Why not process in main thread? (latency, UI blocking)
%% - Sample buffering for AudioWorklet consumption
%% - Synchronization with synthesis pipeline
%%
%% 4.5 Dual-Mode Architecture: Hardware vs Browser Audio
%% - Feature detection: navigator.serial availability
%% - User choice UI: "Use external hardware" toggle
%% - Graceful fallback to getUserMedia
%% - Switching between modes at runtime (if supported)
%%
%% FIGURES:
%% - Data flow diagram: Serial → Main Thread → MessagePort → AudioWorklet
%% - Comparison diagram: Hardware mode vs Browser mode architecture
%%
%% MEASUREMENTS:
%% - JavaScript processing overhead (main thread)
%% - MessagePort transfer latency
%% - Total browser-side latency contribution
%%
%% WRITING TIPS:
%% - Explain the threading model clearly (Serial API = main thread only)
%% - Discuss security implications of Serial API
%% - Show how this integrates with existing AudioWorklet code from Chapter 3
%% - Include error handling strategies

[WRITE YOUR CONTENT HERE]


\section{Performance Analysis and Validation}\label{sec:hw_performance}

%% TODO: Write this section (target: 2-3 pages)
%%
%% This section presents measurements and validates the hardware approach.
%%
%% OUTLINE:
%%
%% 5.1 End-to-End Latency Breakdown
%%
%% Create a detailed latency budget table:
%%
%% \begin{table}[htbp]
%% \centering
%% \caption{Hardware audio acquisition latency breakdown}
%% \label{tab:hw_latency}
%% \begin{tabular}{lrrr}
%% \toprule
%% \textbf{Stage} & \textbf{Typical (ms)} & \textbf{Max (ms)} & \textbf{Notes} \\
%% \midrule
%% MEMS Mic Group Delay    & 0.5  & 0.5  & Datasheet specification \\
%% ES8311 ADC Conversion   & 1.0  & 1.5  & Includes internal buffering \\
%% I2S DMA Buffer          & 11.6 & 11.6 & 512 samples @ 44.1kHz \\
%% ESP32-C3 Processing     & 0.3  & 1.0  & Ring buffer write \\
%% USB Transmission        & 1.0  & 2.0  & Full-speed (1ms frames) \\
%% OS Serial Driver        & 1.5  & 3.0  & Kernel buffering \\
%% Web Serial API          & 2.0  & 4.0  & Browser processing \\
%% MessagePort Transfer    & 0.5  & 1.0  & To AudioWorklet \\
%% \midrule
%% \textbf{Total}          & \textbf{18.4} & \textbf{24.6} & Hardware mode \\
%% \midrule
%% getUserMedia (Browser)  & 55.0 & 85.0 & For comparison \\
%% \bottomrule
%% \end{tabular}
%% \end{table}
%%
%% 5.2 Latency Measurement Methodology
%% - How did you measure each stage?
%% - Tools used: oscilloscope, software timestamps, logic analyzer?
%% - Measurement setup description
%% - Accuracy and limitations of measurements
%%
%% 5.3 Audio Quality Metrics (if measured)
%% - Signal-to-Noise Ratio (SNR): [SPECIFY] dB
%% - Total Harmonic Distortion (THD): [SPECIFY] %
%% - Frequency response (if measured)
%% - Comparison with browser audio quality
%%
%% 5.4 Comparison with Browser-Native Audio
%% - Latency: hardware wins significantly
%% - Consistency: standard deviation across runs
%% - CPU usage: might be higher due to serial overhead?
%% - Audio quality: comparable or better?
%% - Trade-offs: requires hardware, not as universal
%%
%% Create a comparison table or bar chart.
%%
%% 5.5 Power Consumption and Thermal Analysis (optional)
%% - Power draw during idle and active streaming
%% - ESP32-C3 thermal behavior
%% - Battery operation potential (if relevant)
%%
%% 5.6 Reliability and Robustness Testing
%% - Long-duration testing results (hours of continuous operation)
%% - USB re-connection handling
%% - Error recovery testing (cable unplug/replug)
%% - Multi-device testing (if you have multiple units)
%%
%% FIGURES/TABLES REQUIRED:
%% - Table 4.1: Latency breakdown (REQUIRED)
%% - Figure 4.X: Latency comparison bar chart (RECOMMENDED)
%% - Figure 4.X: Oscilloscope trace showing latency measurement (OPTIONAL)
%%
%% WRITING TIPS:
%% - Be honest about measurements: state uncertainty/margin of error
%% - Distinguish between theoretical calculations and actual measurements
%% - Discuss limitations: what couldn't you measure? why?
%% - Compare fairly: acknowledge browser audio advantages (universality, etc.)

[WRITE YOUR CONTENT HERE]

\begin{table}[htbp]
\centering
\caption{Comparison of hardware-accelerated vs browser-native audio acquisition}
\label{tab:hw_comparison}
\begin{tabular}{lcc}
\toprule
\textbf{Metric} & \textbf{Hardware Mode} & \textbf{Browser Mode} \\
\midrule
End-to-end Latency (typ.)  & XX ms & XX ms \\
Latency Std Dev            & XX ms & XX ms \\
CPU Usage (main thread)    & XX\%  & XX\%  \\
Audio SNR                  & XX dB & XX dB \\
Deployment Complexity      & Requires PCB & Universal \\
Cost per Unit              & \$XX  & \$0 \\
Browser Compatibility      & Chrome/Edge only & All browsers \\
\bottomrule
\end{tabular}
\end{table}


\section{Discussion and Limitations}\label{sec:hw_discussion}

%% TODO: Write this section (target: 1-1.5 pages)
%%
%% Be honest about trade-offs and limitations.
%%
%% OUTLINE:
%%
%% 6.1 Advantages of Hardware Approach
%% - Deterministic low latency (<20ms vs 40-80ms)
%% - Consistent performance across different computers
%% - Independence from browser audio stack evolution
%% - Potential for on-board preprocessing (future)
%% - Better control over audio parameters
%%
%% 6.2 Limitations and Trade-offs
%% - Requires physical hardware (cost, availability)
%% - Limited to Chrome/Edge (Web Serial API support)
%% - Additional complexity for end users
%% - Manufacturing and distribution challenges
%% - Not feasible for web-only deployment scenarios
%%
%% 6.3 When to Use Hardware vs Browser Audio?
%% - Hardware mode: professional use, therapy applications, research
%% - Browser mode: casual use, demonstrations, universal access
%% - Recommendation framework based on use case
%%
%% 6.4 Future Hardware Enhancements
%% - On-board pitch detection (offload from browser)
%% - Wireless connectivity (BLE audio?)
%% - Multiple microphone inputs (beamforming?)
%% - Integration with physiological sensors (for healthcare)
%% - Open-source hardware: PCB files, firmware on GitHub
%%
%% WRITING TIPS:
%% - Be balanced: acknowledge both strengths and weaknesses
%% - Don't oversell the hardware (it's not always better)
%% - Provide actionable guidance on when to use which mode
%% - Connect limitations to future work opportunities

[WRITE YOUR CONTENT HERE]

%% =============================================================================
%% END OF CHAPTER
%% =============================================================================
%%
%% FINAL CHECKLIST BEFORE SUBMISSION:
%% □ All sections written to target length
%% □ All figures have captions and are referenced in text
%% □ All tables have captions and are referenced in text
%% □ Code examples are properly formatted with lstlisting
%% □ All measurements include units
%% □ Technical terms are defined on first use
%% □ Consistent terminology with Chapter 3
%% □ All \cite{} references are in mythesis.bib
%% □ Section labels (\label{sec:...}) are used consistently
%% □ Cross-references (\ref{sec:...}) work correctly
%% □ Spelling and grammar checked
%% □ Discussed limitations honestly
%% □ Figures are high quality (300 DPI for photos, vector for diagrams)
%%
%% INTEGRATION CHECKLIST:
%% □ Updated Abstract to mention hardware
%% □ Updated Chapter 3 Section 3.1 to mention dual-mode architecture
%% □ Added hardware evaluation to Chapter 5
%% □ Added hardware references to mythesis.bib
%% □ Updated Conclusion to mention hardware contribution
%% =============================================================================
